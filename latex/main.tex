\documentclass[sigconf]{acmart}

%%
%% \BibTeX command to typeset BibTeX logo in the docs
\AtBeginDocument{%
  \providecommand\BibTeX{{%
    \normalfont B\kern-0.5em{\scshape i\kern-0.25em b}\kern-0.8em\TeX}}}

% %% Rights management information.  This information is sent to you
% %% when you complete the rights form.  These commands have SAMPLE
% %% values in them; it is your responsibility as an author to replace
% %% the commands and values with those provided to you when you
% %% complete the rights form.
% \setcopyright{acmcopyright}
% \copyrightyear{2018}
% \acmYear{2018}
% \acmDOI{10.1145/1122445.1122456}

% %% These commands are for a PROCEEDINGS abstract or paper.
% \acmConference[Woodstock '18]{Woodstock '18: ACM Symposium on Neural
%   Gaze Detection}{June 03--05, 2018}{Woodstock, NY}
% \acmBooktitle{Woodstock '18: ACM Symposium on Neural Gaze Detection,
%   June 03--05, 2018, Woodstock, NY}
% \acmPrice{15.00}
% \acmISBN{978-1-4503-XXXX-X/18/06}

\usepackage{bm}
\usepackage{xspace}
\usepackage[linesnumbered,lined,boxed,commentsnumbered,ruled,vlined]{algorithm2e}
\usepackage{subcaption}
\usepackage{paralist}
\usepackage{multirow}
% \usepackage[table]{xcolor}
\usepackage{colortbl}
\raggedbottom

\def\CI{\mathcal{I}}
\def\CC{\mathcal{C}}
\def\RR{\mathbb{R}}
\def\ZZ{\mathbb{Z}}

% Custom commands for functions and method names
\def\loss{\bm{l}}
\def\true{true}
\def\lab{lab}
\def\data{data}

\newcommand{\pr}[1]{\mbox{Pr}\left\{#1\right\}}

\newcommand{\exsub}[2]{\mbox{E}_{#1}\left\{#2\right\}}
\newcommand{\prsub}[2]{\mbox{Pr}_{#1}\left\{#2\right\}}
\newcommand{\ind}[1]{\mathrm{1}_{#1}}
\newcommand{\prob}[1]{\mbox{Pr}\left\{#1\right\}}
\newcommand{\ex}[1]{\mbox{E}\left\{#1\right\}}

\newcommand{\ignore}[1]{}

\newcommand{\abcomment}[1]{{\color{blue}{[[[AB: #1]]]}}}
\newcommand{\sbcomment}[1]{\textcolor{red}{[[[\textbf{SB: }#1]]]}}

\newcommand{\todo}[1]{{\color{orange} \textbf{TODO:}#1}}
\newcommand{\unsure}[1]{{\color{brown} \textbf{UNSURE:}#1}}
\newcommand{\done}[1]{{\color{teal} \scriptsize \textbf{DONE:}#1}}
\newcommand{\note}[1]{{\color{purple} \scriptsize \textbf{NOTE:}#1}}

\def\vec#1{\boldsymbol{#1}}
\def\vu{\vec{u}}
\def\vv{\vec{v}}
\def\vw{\vec{\omega}}

\def\eucdist#1#2{\varrho(#1, #2)}
\def\projdist#1#2#3{\varrho_{#1}(#2, #3)}
\def\mindist{\varrho_{\min}}
% \def\mag#1{| #1 |}
\def\ip#1#2{\langle #1, #2 \rangle}
\def\RRd#1{\mathbb{R}^{#1}}
\def\1#1{\mathbbm{1}_{#1}}

\def\P#1{\mbox{P}(#1)}
\newcommand\Psub[2]{\mbox{P}_{#1}(#2)}
\def\E#1#2{\mbox{E}_{#1}(#2)}
\def\F#1{\mbox{F}(#1)}
\def\G#1{\mbox{G}(#1)}
\def\H#1{\mbox{H}(#1)}
\def\auc{\texttt{AUC ROC}}

% Move to appendix as appropriate
\newif\ifcutlevone
\newif\ifcutlevtwo
\newif\ifcutlevthree
\cutlevonetrue
\cutlevtwotrue
\cutlevthreefalse

% problem with defining fact
\newtheorem{fact}{Fact}

% \newtheorem{theorem}{Theorem}
% \newtheorem{definition}[theorem]{Definition}
% \newtheorem{fact}{theorem}{Fact}
% \newtheorem{proof}[theorem]{Proof}
% \newtheorem{lemma}[theorem]{Lemma}
% \newtheorem{proposition}[theorem]{Proposition}

% squeezing the life..oops..space out
\setlength{\abovedisplayskip}{1pt}
\setlength{\belowdisplayskip}{1pt}
%\setlength{\abovecaptionskip}{1pt}
%\setlength{\belowcaptionskip}{1pt}
\setlength{\intextsep}{0pt plus 1pt minus 1pt}
\setlength{\textfloatsep}{0pt plus 1.0pt minus 2.0pt}%{10pt plus 1.0pt minus 2.0pt}
\setlength{\dblfloatsep}{-1pt plus 2.0pt minus 2.0pt}
\renewcommand{\floatpagefraction}{1}
% \setlength\parskip{0pt plus 0.1pt minus 0.1pt}
%\raggedbottom

\usepackage{float}

\begin{document}
\title{Benchmarking DeepWalk and Node2Vec}
\author{Anjali, Sumanth Varambally}
\email{ {csz198763,mt6170855}@.iitd.ac.in}
\begin{abstract}
DeepWalk and Node2Vec are were seminal papers in the field of graph representation learning, paving the way for more advanced and contemporary neural-network based architectures like GCNs. In this report, we benchmark Node2Vec and Deepwalk architecture on several prediction tasks such as pair-wise node classification, multi-class node classification, and link prediction. We observe that, as expected, these transductive architectures fail to outperform contemporary inductive architectures. Our code can be found here: \url{https://github.com/anjaliakg17/COL868_Benchmark}
\end{abstract}
\maketitle
% \section{Introduction}

\section{Introduction}
With the recent surge in popularity of Graph Neural Networks, it has become imperative to process graphs and obtain informative representations for nodes and edges. Rather than directly use the vertices and edges of the graphs for inference, we would like to extract semantically useful vector representations for them, which are then further used for downstream tasks like node classification and link prediction. 

There are several aspects involved in extracting semantic features from a graph. Some methods like Node2Vec and DeepWalk aim to produce representations based on structural similarity of nodes. They do so in a transductive manner, i.e. they learn representations for nodes of particular graphs in a task-agnostic manner. However, the model has to be retrained for different graphs, i.e. the representations obtained on one graph are not useful in obtaining representations for other graphs. In comparison, modern methods like GCNs learn representations in an inductive fashion, i.e. model trained on one sets of graphs can be used to extract representations for unseen set of graphs. Typically, the representations so learnt are task-specific, but with a suitable choice of learning objective, task-agnostic representations can be learnt as well.

In this work, we aim to benchmark the Node2Vec and DeepWalk algorithms on the PPI, Protein and Bright Kite datasets. We apply Node2Vec and DeepWalk to these datasets for the tasks of Link Prediction, Pair-wise node classification and multi-class node classification and perform extensive experiments to observe the effects of different parameters on the model performance.

\section{Related Work}
The most commonly used methods for graph-related learning problems are Graph Neural Networks. The idea is to associate each node with a feature vector, and propagate or pass messages across the nodes of the network through the edges. 

Different GNN architectures mainly differ in terms of the aggregation method used to collect and combine feature vectors from neighbourhood nodes \cite{you2019position}.
GCN \cite{Kipf2017SemiSupervisedCW} uses max pooling, GraphSAGE \cite{Hamilton2017InductiveRL} uses mean/max/LSTM pooled information and GAT \cite{velivckovic2017graph} uses transferable attention weights to aggregate information from neighbourhood nodes. PGNNs \cite{you2019position} use sampled anchor nodes to compute the distance of target node from all of these anchor nodes and then learns a non-linear distance-weighted aggregation over the anchor nodes. GraphSAINT \cite{Zeng2020GraphSAINT} uses minibatches that are sampled from the whole graph and then learn full GCNs on each sampled graph. 
\section{Node2Vec}
Node2Vec \cite{grover2016node2vec} uses random walks to sample neighbourhood nodes for a given node, interpolating smoothly between BFS and DFS. Formally, the likelihood of the neighbourhood $N_S(u)$ given the representation $f(u)$ of a given node $u$ is modelled as:
\begin{equation*}
    \Pr(N_S(u)|f(u)) = \prod_{n_i \in N_S(u)} \Pr(n_i|f(u))
\end{equation*}
where the node probability given $f(u)$ is modelled using a softmax function. Further, random walks are simulated with the transition matrix constructed as below:
\begin{equation*}
    \Pr(c_i=x | c_{i-1} = v) = 
    \begin{cases}
    \frac{\pi_{vx}}{Z} \qquad \text{if } (v, x) \in E\\
    0 \quad \qquad \text{otherwise}
    \end{cases}
\end{equation*}
where
$c_i$ is the $i^\text{th}$ node in the walk, $E$ is the set of edges in the graph, $Z$ is a normalizing constant, $\pi_{vx}$ is the un-normalized transition probability between nodes $v$ and $x$. Specific to Node2Vec, there are hyper-parameters $p$ and $q$, which are used to control how fast the walk explores and leaves the node-neighbourhood of starting node $u$. The return parameter $p$ controls the likelihood of revisiting the current node in a walk, and the in-out parameter $q$ allows the search to differentiate between inward and outward nodes. Mathematically, the transition probability $\pi_{vx}$ is modelled as:
\begin{equation*}
    \pi_{vx}(t, x) = 
    \begin{cases}
    \frac{1}{p} \qquad \text{if } d_{tx} = 0\\
    1 \qquad \text{ if } d_{tx} = 1\\
    \frac{1}{q} \qquad \text{if } d_{tx} = 2
    \end{cases}
\end{equation*}
where $d_{tx}$ denotes the shortest path distance between nodes $t$ and $x$.

\section{DeepWalk}
DeepWalk \cite{perozzi2014deepwalk} uses a simlar idea of using random walks to learn feature embeddings. At each vertex, $\gamma$ random walks of length $t$ are simulated, where $\gamma$ and $t$ are fixed. The goal is to maximize the probability of neighbours in the walk, given a vertex representation $f(v_j)$, drawing inspiration from the SkipGram language model. To reduce the complexity of computing the likelihood of a neighbourhood node $u_k$, given a vertex representation $f(v_j)$, the vertices are assigned to the leaves of a binary tree. The prediction problem then turns into maximizing the probability of a specific path in the tree. 

The differences between Node2Vec and DeepWalk are subtle but important. The major difference is the way in which the nodes of the random walk are sampled. In DeepWalk, the next vertex $v_{j+1}$ is picked randomly from among $v_j$'s neighbours, i.e. a first-order Markovian sampling procedure is followed. In Node2Vec, the next node $v_{j+1}$ in the random walk depend on the parameters $p$ and $q$, based on the current node $v_{j}$ and the previous node $v_{j-1}$, i.e. the sampling procedure is second-order Markovian.
 
\section{Experimental Framework}
In this section, we detail the experimental setup used to compare these two methods on various tasks. 

\subsection{Datasets}
We have used the following datasets:
\begin{enumerate}
    \item \textbf{PPI}: The \textbf{P}rotein-\textbf{P}rotein \textbf{I}nteraction dataset \cite{zitnik2017predicting} consists of 24 Protein-Protein Interaction Networks, with each graph having an average of 3000 nodes each and average degree 28.8. The nodes in this network denote different proteins, while the edges represent the interactions between them. Each node may belong to several out of the 121 possible classes. We use this dataset for the task of multi-class node classification and link prediction. 
    
    \item \textbf{Protein}: The Protein dataset \cite{borgwardt2005shortest} consists of 1113 protein graphs, with each node labelled with a functional role of the protein. Each node belongs to one out of three possible classes. We use this dataset for the task of pairwise node-classification
    
    \item \textbf{Brightkite}: Brightkite \cite{cho2011friendship} is a former location-based social networking service where users shared their locations by checking in. We use the collected friendship network, which consists of 58,228 nodes and 214,078 edges, and use it for the task of link prediction. Note here that the task of link prediction has an easily interpretable meaning, i.e. given two people, predict whether they could be friends.
    
\end{enumerate}

\subsection{Task Description}
Here, we detail the different tasks that we have performed during the course of this benchmarking exercise and the data processing steps for each.

\subsubsection{Pair-wise Node Classification}
Given two nodes, we try to predict if they belong to the same class, i.e. they have the same node label. 

We first learn embeddings on the graphs using Node2Vec and DeepWalk in a transductive manner and obtain embeddings for each node. We then train a binary classifier on the aggregated embeddings to predict whether the pair of nodes represented by the aggregated embedding have the same label. We use the Protein dataset for this task, classifying on a random sample of 1,000,000 pairs of nodes. 

\subsubsection{Multi-class Node Classification}
In this task, we assume that a node has multiple class labels associated with it. Given a node, we aim to predict \textit{all} it's associated class labels.

As before, we learn embeddings on the graphs using Node2Vec and DeepWalk in a transductive manner and obtain embeddings for each node. Then, we train a multi-class classifier on the embeddings to predict all the associated class labels. We use the PPI dataset for this task, with each node being associated with multiple of 121 possible class labels.

\subsubsection{Link Prediction}
This task consists of predicting whether there can be an edge between two given nodes. 

For this task, we perform the following procedure:
\begin{enumerate}
    \item Given a graph $G$ with edge set $E$, sample 50\% of the edges to obtain set $E_1$. Also sample `negative` edges $E_{neg}$, i.e. edges between the nodes of G which are not present in $E$. The new training graph will be $G'$ with edge set $E'=E \textbackslash E_1$. 
    \item Learn node embeddings on $G'$. The edges in $G'$ comprise of the `positive` edges in the train set, while a portion of the set $E_{neg}$ is designated as the `negative` edges in the train set.
    \item To construct the test set, we sample positive edges from $E_1$ and negative edges from the remaining portion of $E_{neg}$.
\end{enumerate}
We use both PPI and Brightkite datasets for this task. 

\subsection{Evaluation Metrics}
To evaluate and compare different methods, we use the following evaluation metrics:
\begin{itemize}
    \item \textbf{Precision:} Precision is given by $\left( \frac{\text{True Positives}}{\text{True Positives} + \text{False Positives}}\right)$. It denotes the fraction of the positives predicted that were done so accurately. A low precision score indicates a high false positive rate.
    \item \textbf{Recall:} Recall is given by $\left( \frac{\text{True Positives}}{\text{True Positives} + \text{False Negatives}}\right)$. It denotes the fraction of the true positives in the dataset that were accurately predicted as positives. A low recall score indicates a high false negative rate.
    
    \item \textbf{F1 Score:} A high precision or recall score on it's own does not guarantee good performance since either one of false positives or false negatives might be high. Hence, we use F1 score, which is the harmonic mean of Precision and Recall, to give a more balanced measure of the classifier performance. $\text{F1 Score} = \left( \frac{2}{\frac{1}{\text{Precision}} + \frac{1}{\text{Recall}}}\right)$
    
    \item \textbf{ROC AUC:} The Area Under the Curve of the Receiver Operator Characteristic denotes the area under the true positive rate (vs) false positive rate curve plotted at various thresholds. Therefore, unlike Precision, Recall and F1 score, the ROC AUC score is independent of the threshold at which the classifier is evaluated, and hence can be thought of as a more comprehensive indicator of classifier performance.
\end{itemize}

\subsection{Implementation Details}
In all of our experiments, we used Python 3.7. The implementations of node2vec and DeepWalk that we used were from \url{https://github.com/eliorc/node2vec} and \url{https://github.com/phanein/deepwalk} respectively. All experiments were run on CPU in Google Colab, with a single core Xeon CPU with 2 threads.

We perform extensive hyperparameter tuning to determine the effect of each parameter on model performance, and to determine the best performing set of parameters. The range of hyperparameters explored is presented in Table \ref{tab:parameter_ranges}. For Pair-wise Node Classification and Link Prediction, we experimented with different aggregation functions to obtain a single embedding; the details of which we have summarized in Table \ref{tab:aggregation_functions}. 

\begin{table}[H]
    \centering
    \begin{tabular}{|c|c|}
        \hline
        \textbf{Parameters} & \textbf{Set of values} \\
        \hline
         Embedding dimension & 32, 64, 128\\
         Walk Length & 10, 40, 100 \\
         Number of Walks & 10, 50, 100\\
         Return Parameter $p$ & 0.2, 1, 2\\
         In-Out Parameter $q$ & 0.2, 1, 2\\
         Window Size & 5, 10, 20\\
         \hline
    \end{tabular}
    \caption{Hyper-Parameter Ranges for DeepWalk and Node2Vec}
    \label{tab:parameter_ranges}
\end{table}


\begin{table}[H]
\centering
\begin{tabular}{|c|c|c|}
    \hline
    \textbf{Operator} & \textbf{Symbol} & \textbf{Definition}\\
    \hline
     Average & $\boxplus$ & $f(u) \boxplus f(v) = \frac{f(u)+f(v)}{2}$\\
     Hadamard & $\boxtimes$ & $f(u) \boxtimes f(v) = \left[\frac{f_i(u)\times f_i(v)}{2}\right]_{i=1}^n$\\
     L1 & $\left\lVert\cdot\right\rVert_1$ & $\left\lVert f(u) - f(v) \right \rVert_1 = \left[|f_i(u)-f_i(v)|\right]_{i=1}^n$\\
     L2 & $\left\lVert\cdot\right\rVert_2$ & $\left\lVert f(u) - f(v) \right \rVert_2 = \left[|f_i(u)-f_i(v)|^2\right]_{i=1}^n$\\
     \hline
\end{tabular}

     \caption{Choice of aggregation functions}
     \label{tab:aggregation_functions}
\end{table}
We have used the Logistic Regression classifier for the Link Prediction task, following the setup detailed in \cite{grover2016node2vec} and the Random Forest Classifier for the Multi-Class Classification and Pair-wise Node Classification tasks. All the classifiers were from the sklearn package. For the Multi-Class classification task, we used the Multi-Output Classifier class along with Random Forest from the sklearn package. We use 5-fold cross validation for the Node Classification tasks, and a 80-20\% split for Link Prediction task.

\section{Results}
We report the result for each task and dataset here. In each case, the optimal parameters were obtained through extensive grid search over different ranges of parameter settings.
\subsection{Pairwise Node Classification Task}
We detail the results of the pair-wise node classification task in Table \ref{tab:pairwise_node_classification}. As can be seen from the table, Node2Vec and DeepWalk perform very poorly, especially in comparison to GNN based methods like P-GNN and GAT. This can be attributed to the fact that these methods can exploit node features better, and also because they can be trained in a task-specific manner, whereas the node embeddings learnt by DeepWalk and Node2Vec are task-agnostic.
\begin{table}[H]
    \centering
    \begin{tabular}{|c|c|c|c|c|}
    \hline
        \textbf{Method} & \textbf{Precision} & \textbf{Recall} & \textbf{F1 score} & \textbf{ROC AUC} \\
        \hline
        P-GNN & 0.811 &	0.803 &	0.576 &	0.647\\
        GCN	&	0.495 & 0.506 & 0.506 & 0.506\\
        GraphSAGE &	0.499 & 0.5 & 0.5 & 0.5\\
        GAT	&	\textbf{0.896} & \textbf{0.829} & \textbf{0.829} & \textbf{0.829} \\
        \hline
         Node2Vec & 0.513 & 0.513 & 0.513 & 0.502\\
         DeepWalk & 0.509 & 0.509 & 0.509 & 0.504\\
         \hline
    \end{tabular}
    \caption{Performance of different methods on Pairwise Node Classification on Protein Dataset}
    \label{tab:pairwise_node_classification}
\end{table}

\subsection{Link Prediction}
Tables \ref{tab:link_prediction_ppi} and \ref{tab:link_prediction_brightkite} respectively show the results of DeepWalk and Node2Vec on PPI and BrightKite datasets. It is surprising to observe that these two simple architectures perform so well, sometimes even outperforming the GNN architectures. We attribute this to two possible reasons - (a) our testing methodology might be slightly different to ones used in other methods (b) the transductive nature of these algorithms might lend them an upper hand in this task. It is also surprising that DeepWalk outperforms Node2Vec on the Brightkite dataset, but this could be due to suboptimal parameter choices for Node2Vec. 
\begin{table}[H]
    \centering
    \begin{tabular}{|c|c|c|c|c|}
    \hline
        \textbf{Method} & \textbf{Precision} & \textbf{Recall} & \textbf{F1 score} & \textbf{ROC AUC} \\
        \hline
        GCN	&	0.772&	0.655&	0.655&	0.655\\
        GraphSAGE	&	0.790&	0.667&	0.667&	0.667\\
        GAT & 0.902&	0.822&	0.822&	0.822\\
        P-GNN & 0.784	& 0.730 &	0.701	& 0.715\\
        \hline
         Node2Vec & \textbf{0.957} & 0.929 & 0.943 & 0.944\\
         DeepWalk & 0.956 & \textbf{0.989} & \textbf{0.972} & \textbf{0.972}\\
         \hline
    \end{tabular}
    \caption{Performance of different methods on Link Prediction on PPI Dataset}
    \label{tab:link_prediction_ppi}
\end{table}

\begin{table}[H]
    \centering
    \begin{tabular}{|c|c|c|c|c|}
    \hline
        \textbf{Method} & \textbf{Precision} & \textbf{Recall} & \textbf{F1 score} & \textbf{ROC AUC} \\
        \hline
        P-GNN & 0.795	& 0.751	& 0.694& 	0.721\\
        GAT	& 0.891 &	\textbf{0.806}&	0.806&	0.806\\
        GCN	& 0.925	& -	& -	& - \\ 	
        GraphSAGE	& 0.934 & -&-&-\\	
        \hline
         Node2Vec & 0.670 & 0.770 & 0.717 & 0.696\\
         DeepWalk & \textbf{0.963} & 0.803 & \textbf{0.876} & \textbf{0.886}\\
         \hline
    \end{tabular}
    \caption{Performance of different methods on Link Prediction on Brightkite Dataset}
    \label{tab:link_prediction_brightkite}
\end{table}

\subsection{Multi-Class Classification}
The results for multi-class classification are presented in Table \ref{tab:multiclass_classification_ppi}. As expected, while the two transductive methods perform decently, their results wane in comparison to the GNN based architectures, especially GAT. Again, this could be attributed to their ability to exploit node features as well as their specificity.
\begin{table}[H]
    \centering
    \begin{tabular}{|c|c|c|c|c|}
    \hline
        \textbf{Method} & \textbf{Precision} & \textbf{Recall} & \textbf{F1 score} & \textbf{ROC AUC} \\
        \hline
        PGNN	&	0.656&	0.621&	0.481&	0.543\\
        GCN	&	0.789&0.818&0.744&0.780\\
GraphSAGE	&	0.960	&\textbf{0.957}	&\textbf{0.953}&	\textbf{0.955}\\
GAT	&	\textbf{0.97}&	0.924&	0.925&	0.925\\
        \hline
         Node2Vec & 0.634 & 0.463 & 0.536 & 0.673\\
         DeepWalk & 0.643 & 0.430 & 0.515 & 0.662\\
         \hline
    \end{tabular}
    \caption{Performance of different methods on Multi-Class Classification on PPI Dataset}
    \label{tab:multiclass_classification_ppi}
\end{table}

\subsection{Effect of Hyperparameters}
We look at the effect and trends observed when different hyperparameters are varied. For this purpose, we evaluate different hyperparameter settings on the Multi-Class classification task on the PPI dataset.

\subsubsection{Embedding Dimension}
Table \ref{tab:embed_dim} shows the variation of performance of DeepWalk with Embedding dimensions, with all other parameters kept fixed. We observe no discernable pattern or trend in the performance as the embedding dimension is changed. We would like to mention here that the original paper reported a slight increase in performance with increase in embedding dimension, but our results suggest that the correlation is tenuous at best.
\begin{table}[H]
\begin{tabular}{|r|r|r|r|r|}
\hline
\textbf{Embed. Dimension} & \textbf{ROC} & \textbf{Precision} & \textbf{Recall} & \textbf{F1 Score}\\
\hline
32 & 0.649 & 0.655 & 0.387 & 0.486\\
64 & 0.646 & 0.650 & 0.383 & 0.482\\
128 & 0.647 & 0.648 & 0.387 & 0.484\\
\hline
\end{tabular}

\caption{Effect of Embedding Dimension on Node2Vec performance in Multi-Class Classification}
\label{tab:embed_dim}
\end{table}


\subsubsection{Walk Length}
Tables \ref{tab:walk_length_n2v} and 
\ref{tab:walk_length_dw} show the variation of performance of Node2Vec and DeepWalk with walk length, with all other parameters kept fixed. We observe no discernable pattern with Node2Vec, while DeepWalk experiences a slight increase in performance. It is worth mentioning that the original papers report a significant increase in performance with increased walk length, which we were unable to replicate.

\begin{table}[H]
\begin{tabular}{|r|r|r|r|r|}
\hline
\textbf{Walk Length} & \textbf{ROC} & \textbf{Precision} & \textbf{Recall} & \textbf{F1 Score}\\
\hline
10 & 0.641 & 0.634 & 0.377 & 0.473\\
40 & 0.646 & 0.650 & 0.382 & 0.482\\
100 & 0.626 & 0.641 & 0.334 & 0.440\\
\hline
\end{tabular}
\caption{Effect of Walk Length on Node2Vec performance in Multi-Class Classification}
\label{tab:walk_length_n2v}
\end{table}

\begin{table}[H]
\begin{tabular}{|r|r|r|r|r|}
\hline
\textbf{Walk Length} & \textbf{ROC} & \textbf{Precision} & \textbf{Recall} & \textbf{F1 Score}\\
\hline
10 & 0.637 & 0.624 & 0.374 & 0.468\\
80 & 0.648 & 0.639 & 0.394 & 0.487\\
200 & 0.650 & 0.641 & 0.399 & 0.491\\
\hline
\end{tabular}
\caption{Effect of Walk Length on DeepWalk performance in Multi-Class Classification}

\label{tab:walk_length_dw}
\end{table}

\subsubsection{Number of Walks}
Tables \ref{tab:nwalk_n2v} and 
\ref{tab:nwalk_dw} show the variation of performance of Node2Vec and DeepWalk with number of walks, with all other parameters kept fixed. A similar observation as with walk length follows, with the increase in performance reported in the original paper unable to be replicated.
\begin{table}[H]
\begin{tabular}{|r|r|r|r|r|}
\hline
\textbf{\#Walks} & \textbf{ROC} & \textbf{Precision} & \textbf{Recall} & \textbf{F1 Score}\\
\hline
10 & 0.646 & 0.650 & 0.382 & 0.482\\
20 & 0.628 & 0.642 & 0.341 & 0.445\\
50 & 0.623 & 0.641 & 0.327 & 0.433\\
\hline
\end{tabular}
\caption{Effect of Number of Walks on Node2Vec performance in Multi-Class Classification}
\label{tab:nwalk_n2v}
\end{table}

\begin{table}[H]
\begin{tabular}{|r|r|r|r|r|}
\hline
\textbf{\# Walks} & \textbf{ROC} & \textbf{Precision} & \textbf{Recall} & \textbf{F1 Score}\\
\hline
5 & 0.645 & 0.638 & 0.388 & 0.482\\
10 & 0.648 & 0.640 & 0.394 & 0.488\\
50 & 0.651 & 0.641 & 0.401 & 0.493\\
\hline
\end{tabular}
\caption{Effect of Number of Walks on DeepWalk performance in Multi-Class Classification}
\label{tab:nwalk_dw}
\end{table}


\subsubsection{Parameters $p$ and $q$}
Tables \ref{tab:p} and \ref{tab:q} represent the trends in performance of Node2Vec with change in parameters $p$ and $q$ respectively. Although we observe an increase in performance with increased $p$ and decreased $q$, we must note here that these parameter depend heavily on the dataset itself and hence a monotonic trend as observed here might not always be reflected in practice.
\begin{table}[H]
\begin{tabular}{|r|r|r|r|r|}
\hline
\textbf{$p$} & \textbf{ROC} & \textbf{Precision} & \textbf{Recall} & \textbf{F1 Score}\\
\hline
0.2 & 0.637 & 0.645 & 0.362 & 0.464\\
1 & 0.646 & 0.650 & 0.382 & 0.481\\
2 & 0.648 & 0.652 & 0.388 & 0.486\\
\hline
\end{tabular}
\caption{Effect of $p$ on Node2Vec performance in Multi-Class Classification}
\label{tab:p}
\end{table}

\begin{table}[H]
\begin{tabular}{|r|r|r|r|r|}
\hline
\textbf{$q$} & \textbf{ROC} & \textbf{Precision} & \textbf{Recall} & \textbf{F1 Score}\\
\hline
0.2 & 0.648 & 0.646 & 0.391 & 0.487\\
1 & 0.646 & 0.649 & 0.383 & 0.481\\
2 & 0.643 & 0.649 & 0.376 & 0.476\\
\hline
\end{tabular}
\caption{Effect of $q$ on Node2Vec performance in Multi-Class Classification}
\label{tab:q}
\end{table}

\subsubsection{Window Size}
Table \ref{tab:window_size} represents the variation in performance of DeepWalk with change in window-size. As expected, a slight increase in performance is observed with increase in the window size.
\begin{table}[H]
\begin{tabular}{|r|r|r|r|r|}
\hline
\textbf{Window Size} & \textbf{ROC} & \textbf{Precision} & \textbf{Recall} & \textbf{F1 Score}\\
\hline
5 & 0.641 & 0.636 & 0.378 & 0.474\\
10 & 0.648 & 0.640 & 0.393 & 0.488\\
20 & 0.651 & 0.641 & 0.401 & 0.494\\
\hline
\end{tabular}
\caption{Effect of Window Size on DeepWalk performance in Multi-Class Classification}
\label{tab:window_size}
\end{table}

\subsubsection{Effect of Aggregation function on Link Prediction}

Tables \ref{tab:agg_n2v_ppi} and \ref{tab:agg_dw_ppi} show the performance of Node2Vec and DeepWalk respectively on the PPI dataset, while tables \ref{tab:agg_n2v_brightkite} and \ref{tab:agg_dw_brightkite} show the performance of Node2Vec and DeepWalk respectively on the Brightkite dataset. A significant difference in performance is observed across the different aggregation functions, especially across different architectures and datasets. The Hadamard aggregation function performs well on the PPI dataset, but performs poorly on the Brightkite dataset. However, the L1 and L2 aggregation functions show roughly the same performance, and are the better option in most scenarios.
\begin{table}[H]
\begin{tabular}{|r|r|r|r|r|}
\hline
\textbf{Agg. Function} & \textbf{ROC} & \textbf{Precision} & \textbf{Recall} & \textbf{F1 Score}\\
\hline
Average & 0.725 & 0.711 & 0.761 & 0.735\\
Hadamard & 0.923 & 0.985 & 0.860 & 0.918\\
L1 & 0.933 & 0.957 & 0.906 & 0.931\\
L2 & 0.944 & 0.957 & 0.929 & 0.943\\
\hline
\end{tabular}
\caption{Performance of Node2Vec on PPI dataset in Link Prediction using different aggregation functions}
\label{tab:agg_n2v_ppi}
\end{table}


\begin{table}[H]
\begin{tabular}{|r|r|r|r|r|}
\hline
\textbf{Agg. Function} & \textbf{ROC} & \textbf{Precision} & \textbf{Recall} & \textbf{F1 Score}\\
\hline
Average & 0.685 & 0.666 & 0.742 & 0.702\\
Hadamard & 0.972 & 0.956 & 0.989 & 0.972\\
L1 & 0.946 & 0.977 & 0.914 & 0.944\\
L2 & 0.950 & 0.976 & 0.922 & 0.948\\
\hline
\end{tabular}
\caption{Performance of DeepWalk on PPI dataset in Link Prediction using different aggregation functions}
\label{tab:agg_dw_ppi}
\end{table}

\begin{table}[H]
\begin{tabular}{|r|r|r|r|r|}
\hline
\textbf{Agg. Function} & \textbf{ROC} & \textbf{Precision} & \textbf{Recall} & \textbf{F1 Score}\\
\hline
Average & 0.696   & 0.670 & 0.770 & 0.717   \\
Hadamard & 0.752 & 0.927 & 0.546 & 0.687   \\
L1 & 0.748 & 0.987 & 0.503 & 0.667   \\
L2 & 0.749 & 0.987 & 0.504 & 0.667 \\
\hline
\end{tabular}
\caption{Performance of Node2Vec on Brightkite dataset in Link Prediction using different aggregation functions}
\label{tab:agg_n2v_brightkite}
\end{table}


\begin{table}[H]
\begin{tabular}{|r|r|r|r|r|}
\hline
\textbf{Agg. Function} & \textbf{ROC} & \textbf{Precision} & \textbf{Recall} & \textbf{F1 Score}\\
\hline
Average & 0.750 & 0.707 & 0.856 & 0.774  \\
Hadamard & 0.716 & 0.807 & 0.567 & 0.666   \\
L1 & 0.883 & 0.967 & 0.793 & 0.871   \\
L2 & 0.886 & 0.963 & 0.803 & 0.876 \\
\hline
\end{tabular}
\caption{Performance of DeepWalk on Brightkite dataset in Link Prediction using different aggregation functions}
\label{tab:agg_dw_brightkite}
\end{table}


\subsection{Timing Analysis}
Tables \ref{tab:timing_n2v} and \ref{tab:timing_dw} show the time taken by the Node2Vec and DeepWalk architectures for learning node embeddings for the Link Prediction task on the PPI dataset. Since the most time-sensitive hyper-parameter in both these methods is the number of walks, we show the train time for different number of walks and the associated performance. As the tables indicate, a higher train time need not always indicate better performance (as evidenced by the study of the impact of the number of walks on performance earlier). However, a striking observation is that DeepWalk is significantly faster than Node2Vec, with negligable difference in performance. We attribute this to the binary tree optimization performed by the DeepWalk implementation for improved runtime.

\begin{table}[H]
\begin{tabular}{|r|r|r|r|r|r|}
\hline
\textbf{\#Walks} & \textbf{ROC} & \textbf{Precision} & \textbf{Recall} & \textbf{F1 Score} & \textbf{Train Time (s)}\\
\hline
2&	0.965&	0.943&	0.990&	0.966&	516\\
10&	0.968&	0.953&	0.985&	0.969&	1865\\
20&	0.956&	0.955&	0.958&	0.956&	3682\\
40&	0.943&	0.957&	0.927&	0.942&	7216\\
\hline
\end{tabular}
\caption{Timing Analysis of Node2Vec on Link Prediction task on PPI dataset}
\label{tab:timing_n2v}
\end{table}

\begin{table}[H]
\begin{tabular}{|r|r|r|r|r|r|}
\hline
\textbf{\#Walks} & \textbf{ROC} & \textbf{Precision} & \textbf{Recall} & \textbf{F1 Score} & \textbf{Train Time (s)}\\
\hline
2&0.937&	0.932&	0.943&	0.937&	69\\
5&0.955&	0.965&	0.943&	0.954&	145\\
10&0.952&	0.971&	0.931&	0.951&	268\\
20&0.950&	0.974&	0.924&	0.948&	507\\
40&0.950&	0.975&	0.925&	0.949&	977\\
\hline
\end{tabular}
\caption{Timing Analysis of DeepWalk on Link Prediction task on PPI dataset}

\label{tab:timing_dw}
\end{table}

\section{Conclusions}
We make the following concluding observations:
\begin{itemize}
    \item In most use-cases, contemporary inductive architectures like GCNs outperform Node2Vec and DeepWalk.
    \item DeepWalk and Node2Vec are completely unsuited for datasets (like Proteins) where node labels are important informative features. On the other hand, they might work better in scenarios where structural properties are more important (like in link prediction).
    \item The trends reported in the original papers of the performance of these models with different hyperparameters, are not readily reproducible, and should be taken with a grain of salt.
    \item It is not necessarily the case that Node2Vec outperforms DeepWalk in all settings, however in most cases the difference in performance is marginal at best.
    \item In most time-sensitive applications, DeepWalk seems to be a better choice than Node2Vec. Note, however, that this could be an implementation issue. We see no reason that the tree-based optimization cannot be applied on Node2Vec.
\end{itemize}

\bibliographystyle{ACM-Reference-Format}
\bibliography{main}
\pagebreak\clearpage

\end{document}
